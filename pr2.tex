\documentclass[a4paper,12pt]{book}
\usepackage{etex}
\usepackage[utf8]{inputenc}
\usepackage[T1]{fontenc}
\usepackage{fullpage}
\usepackage{amsmath}
\usepackage{amsthm}
\usepackage{txfonts}
\usepackage{latexsym}
\usepackage{stmaryrd}
\usepackage{amssymb}
\usepackage{mathrsfs}
%\usepackage[all]{xy}
\usepackage{proof}
\usepackage[sans]{dsfont}
\usepackage[spanish]{babel}
%%% Lógica
\def\limp{\Rightarrow}
\def\Ra{\Rightarrow}
\def\liff{\Leftrightarrow}
\def\ForIn#1#2{(\forall#1\,{\in}\,#2)}
\def\ExIn#1#2{(\exists#1\,{\in}\,#2)}
\def\ForSub#1#2{(\forall#1\,{\subseteq}\,#2)}
\def\ExSub#1#2{(\exists#1\,{\subseteq}\,#2)}
%%% Conjuntos
\def\N{\mathds{N}}
\def\Z{\mathds{Z}}
\def\Q{\mathds{Q}}
\def\R{\mathds{R}}
\def\Pow{\mathfrak{P}}
\def\id{\mathrm{id}}
\def\dom{\mathrm{dom}}
\def\img{\mathrm{img}}
\def\pto{\rightharpoonup}
\def\into{\hookrightarrow}
\def\onto{\twoheadrightarrow}
\def\inonto{\mathbin{\tilde{\to}}}
%%% Órdenes
\def\A{\mathcal{A}}
\def\B{\mathcal{B}}
\def\C{\mathcal{C}}
\def\I{\mathcal{I}}
\def\Seg{\mathit{Seg}}
%%% Misc.
\def\Cl{\mathit{Cl}}
\def\On{\mathit{On}}
\def\ForOn#1{(\forall#1\,{:}\,\On)}
\def\ExOn#1{(\exists#1\,{:}\,\On)}
\def\ds{\displaystyle}
\def\FV{\mathit{FV}}
%%%
\theoremstyle{definition}
\newtheorem{ejercicio}{Ejercicio}
\outer\long\def\COUIC#1{}
\outer\long\def\Solucion#1{\par
  {\sl\small\noindent\textbf{Solución:}\quad#1\par}}
%%% Comentar la siguiente línea para mostrar las soluciones
\outer\long\def\Solucion#1{}

\begin{document}

\noindent
\centerline{\sc
Facultad de Ciencias\hfill---\hfill
Fundamentos de la Matemática\hfill---\hfill
Segundo semestre de 2018}\smallbreak\hrule

\bigbreak
\centerline{\Large\textbf{Práctico 2: Deducción natural y teorías}}

\bigbreak
\paragraph{Definiciones}
Sea $\mathcal{V}$ el vocabulario de un lenguaje de primer orden.
\begin{itemize}
\item Los \emph{términos} (notación: $t$, $u$, $v$, etc.) están
  definidos por la gramática:
  $$t,u,v~::=~x~~|~~ f(t_1,\ldots,t_k)$$
  (donde $f$ es cualquier símbolo de función de aridad~$k$ en el
  vocabulario~$\mathcal{V}$).
\item Las \emph{fórmulas} (notación: $\phi$, $\psi$, $\chi$, etc.)
  están definidas por la gramática:
  $$\begin{array}{rrl}
    \phi,\psi,\chi&::=&\top~~|~~\bot~~|~~ t_1=t_2
    ~~|~~ p(t_1,\ldots,t_k)\\
    &|&\lnot\phi~~|~~\phi\land\psi~~|~~\phi\lor\psi
    ~~|~~\phi\limp\psi~~|~~\phi\liff\psi\\
    &|&\forall x\,\phi~~|~~\exists x\,\phi\\
  \end{array}\eqno\begin{tabular}{r@{}}
  (Fórmulas atómicas)\\
  (Conectivas)\\
  (Cuantificadores)\\
  \end{tabular}$$
  (donde $p$ es cualquier símbolo de predicado de aridad~$k$ en el
  vocabulario~$\mathcal{V}$).
\item Los \emph{contextos} (notación: $\Gamma$, $\Delta$, etc.)
  son los conjuntos finitos de fórmulas:
  $$\Gamma,\Delta~::=~\phi_1,\ldots,\phi_n\eqno(n\ge 0)$$
\item Un \emph{secuente} es un par escrito $\Gamma\vdash\phi$
  («$\phi$ es consecuencia de~$\Gamma$», o «$\Gamma$
  demuestra~$\phi$»), donde~$\Gamma$ es un contexto (el
  \emph{antecedente}, cuyos elementos se llaman las \emph{hipótesis}
  del secuente) y $\phi$ una fórmula  
  (el \emph{consecuente}, o la \emph{conclusión} del secuente).
\end{itemize}
Las reglas de deducción del sistema NK («deducción natural clásica»)
son las siguientes:
$$\def\myskip{\medskip}
\begin{array}{lc@{\qquad}c}
  (\text{Axioma}) & \multicolumn{2}{c}{
    \infer[\text{\small(si~$\phi\in\Gamma$)}]
    {\Gamma\vdash\phi}{}
  }\\
  \noalign{\myskip}
  ({\top},{\bot}) & \begin{array}{c}
    \infer{\Gamma\vdash\top}{\mathstrut}
  \end{array} & \begin{array}{c}
    \infer{\Gamma\vdash\phi}{\Gamma,\lnot\phi\vdash\bot}
  \end{array}\rlap{\quad(Absurdo)} \\
  \noalign{\myskip}
  ({\lnot}) & \begin{array}{c}
    \infer{\Gamma\vdash\lnot\phi}{\Gamma,\phi\vdash\bot}
  \end{array} & \begin{array}{c}
    \infer{\Gamma\vdash\bot}
    {\Gamma\vdash\phi & \Gamma\vdash\lnot\phi}
  \end{array} \\
  \noalign{\myskip}
  ({\land}) & \begin{array}{c}
    \infer{\Gamma\vdash\phi\land\psi}
    {\Gamma\vdash\phi & \Gamma\vdash\psi}
  \end{array} & \begin{array}{c}
    \infer{\Gamma\vdash\phi}{\Gamma\vdash\phi\land\psi}\qquad
    \infer{\Gamma\vdash\psi}{\Gamma\vdash\phi\land\psi}
  \end{array} \\
  \noalign{\myskip}
  ({\lor}) & \begin{array}{c}
    \infer{\Gamma\vdash\phi\lor\psi}{\Gamma\vdash\phi}\qquad
    \infer{\Gamma\vdash\phi\lor\psi}{\Gamma\vdash\psi}
  \end{array} & \begin{array}{c}
    \infer{\Gamma\vdash\chi}
    {\Gamma\vdash\phi\lor\psi &
      \Gamma,\phi\vdash\chi & \Gamma,\psi\vdash\chi}
  \end{array} \\
  \noalign{\myskip}
  ({\limp}) & \begin{array}{c}
    \infer{\Gamma\vdash\phi\limp\psi}{\Gamma,\phi\vdash\psi}
  \end{array} & \begin{array}{c}
    \infer{\Gamma\vdash\psi}
    {\Gamma\vdash\phi\limp\psi & \Gamma\vdash\phi}
  \end{array} \\
  \noalign{\myskip}
  ({\liff}) & \begin{array}{c}
    \infer{\Gamma\vdash\phi\liff\psi}{
      \Gamma,\phi\vdash\psi & \Gamma,\psi\vdash\phi}
  \end{array} & \begin{array}{c}
    \infer{\Gamma\vdash\psi}
    {\Gamma\vdash\phi\liff\psi & \Gamma\vdash\phi}\qquad
    \infer{\Gamma\vdash\phi}
    {\Gamma\vdash\phi\liff\psi & \Gamma\vdash\psi}
  \end{array} \\
  \noalign{\myskip}
  ({\forall}) & \begin{array}{c}
    \infer[\text{\small(si $x\notin\FV(\Gamma)$)}]
    {\Gamma\vdash\forall x\,\phi}{\Gamma\vdash\phi}
  \end{array} & \begin{array}{c}
    \infer{\Gamma\vdash\phi[x:=t]}{\Gamma\vdash\forall x\,\phi}
  \end{array} \\
  \noalign{\myskip}
  ({\exists}) & \begin{array}{c}
    \infer{\Gamma\vdash\exists x\,\phi}{\Gamma\vdash\phi[x:=t]}
  \end{array} & \begin{array}{c}
    \infer[\text{\small(si $x\notin\FV(\Gamma,\psi)$)}]
    {\Gamma\vdash\psi}
    {\Gamma\vdash\exists x\,\phi & \Gamma,\phi\vdash\psi}
  \end{array} \\
  \noalign{\myskip}
  ({=}) & \begin{array}{c}
    \infer{\Gamma\vdash t=t}{\mathstrut}
  \end{array} & \begin{array}{c}
    \infer{\Gamma\vdash\phi[x:=u]}
    {\Gamma\vdash t=u & \Gamma\vdash\phi[x:=t]}
  \end{array} \\
\end{array}$$

\newpage
\paragraph{Reglas admisibles}\label{p:ReglasAdmisibles}
Dados secuentes $S_{\!1},\ldots,S_{\!p}$ y $S$, se dice que la regla
\begin{center}
  $\infer={S}{S_{\!1}~\cdots~S_{\!p}}$
\end{center}
es \emph{admisible} (en un sistema de deducción cualquiera) cuando
cumple la propiedad que:
\begin{center}
  Si los secuentes $S_1,\ldots,S_n$ son derivables, entonces
  el secuente~$S$ es derivable.
\end{center}
Es claro que toda regla admisible (indicada tradicionalmente con
una doble raya de inferencia) se puede agregar al sistema de
deducción sin cambiar la clase de los secuentes derivables.

En lo siguiente, se autorizará el uso de las reglas admisibles presentadas en el teórico.

\bigbreak
\begin{ejercicio}
  --- Derivar las siguientes fórmulas (como secuentes con contexto vacío):
  \begin{itemize}
  \item Equivalencias notables del cálculo proposicional:\\[3pt]
    $\begin{array}{@{}l@{\qquad}l@{}}
      \phi\land\phi\liff\phi&
      \phi\lor\phi\liff\phi\\
      \phi\land\psi\liff\psi\land\phi&
      \phi\lor\psi\liff\psi\lor\phi\\
      (\phi\land\psi)\land\chi\liff\phi\land(\psi\land\chi)&
      (\phi\lor\psi)\lor\chi\liff\phi\lor(\psi\lor\chi)\\
      \phi\land\top\liff\phi&
      \phi\lor\bot\liff\phi\\
      \phi\land\bot\liff\bot&
      \phi\lor\top\liff\top\\
      \phi\land(\psi\lor\chi)\liff(\phi\land\psi)\lor(\phi\land\chi)&
      \phi\lor(\psi\land\chi)\liff(\phi\lor\psi)\land(\phi\lor\chi)\\
      (\phi\limp\psi\land\chi)\liff(\phi\limp\psi)\land(\phi\limp\chi)&
      (\phi\limp\psi\lor\chi)\liff(\phi\limp\psi)\lor(\phi\limp\chi)\\
      (\phi\land\psi\limp\chi)\liff(\phi\limp\chi)\lor(\psi\limp\chi)&
      (\phi\lor\psi\limp\chi)\liff(\phi\limp\chi)\land(\psi\limp\chi)\\
      (\top\limp\phi)\liff\phi&(\phi\limp\bot)\liff\lnot\phi\\
      (\bot\limp\phi)\liff\top&(\phi\limp\top)\liff\top\\
      (\phi\limp(\psi\limp\chi))\liff(\phi\land\psi\limp\chi)\\
    \end{array}$
  \item Equivalencias notables del cálculo de predicados:\\[3pt]
    $\begin{array}{@{}ll@{\qquad}ll@{}}
      \forall x\,(\phi\land\psi)\liff
      \forall x\,\phi\land\forall x\,\psi&&
      \exists x\,(\phi\lor\psi)\liff
      \exists x\,\phi\lor\exists x\,\psi\\
      \forall x\,\phi\lor\forall x\,\psi\limp\forall x\,(\phi\lor\psi)&&
      \exists x\,(\phi\land\psi)\limp
      \exists x\,\phi\land\exists x\,\psi\\
      \forall x\,\phi\liff\phi&
      \text{\footnotesize(si $x\notin\FV(\phi)$)}&
      \exists x\,\phi\liff\phi&
      \text{\footnotesize(si $x\notin\FV(\phi)$)}\\
      \forall x\,(\phi\land\psi)\liff\phi\land\forall x\,\psi&
      \text{\footnotesize(si $x\notin\FV(\phi)$)}&
      \exists x\,(\phi\lor\psi)\liff\phi\lor\exists x\,\psi&
      \text{\footnotesize(si $x\notin\FV(\phi)$)}\\
      \forall x\,(\phi\lor\psi)\liff\phi\lor\forall x\,\psi&
      \text{\footnotesize(si $x\notin\FV(\phi)$)}&
      \exists x\,(\phi\land\psi)\liff\phi\land\exists x\,\psi&
      \text{\footnotesize(si $x\notin\FV(\phi)$)}\\
      \forall x\,(\phi\limp\psi)\liff(\phi\limp\forall x\,\psi)&
      \text{\footnotesize(si $x\notin\FV(\phi)$)}&
      \exists x\,(\phi\limp\psi)\liff(\phi\limp\exists x\,\psi)&
      \text{\footnotesize(si $x\notin\FV(\phi)$)}\\
      \forall x\,(\phi\limp\psi)\liff(\exists x\,\phi\limp\psi)&
      \text{\footnotesize(si $x\notin\FV(\psi)$)}&
      \exists x\,(\phi\limp\psi)\liff(\forall x\,\phi\limp\psi)&
      \text{\footnotesize(si $x\notin\FV(\psi)$)}\\
    \end{array}$
  \item Tautologías clásicas y leyes de De Morgan:\\[3pt]
    $\begin{array}{l@{\qquad}l}
      \lnot\lnot\phi\liff\phi&
      \phi\lor\lnot\phi\\
      \lnot\top\liff\bot&
      \lnot\bot\liff\top\\
      \lnot(\phi\land\psi)\liff\lnot\phi\lor\lnot\psi&
      \lnot(\phi\lor\psi)\liff\lnot\phi\land\lnot\psi\\
      \lnot(\phi\limp\psi)\liff\phi\land\lnot\psi&
      (\phi\limp\psi)\liff\lnot\phi\lor\psi\\
      \lnot\forall x\,\phi\liff\exists x\,\lnot\phi&
      \lnot\exists x\,\phi\liff\forall x\,\lnot\phi\\
      ((\phi\limp\psi)\limp\phi)\limp\phi &
      \text{(ley de Peirce)} \\
    \end{array}$
    \newpage
  \item Igualdad:\\[3pt]
    $\begin{array}{ll}
      \forall x\,(x=x) \\
      \forall x\,\forall y\,(x=y\limp y=x)\\
      \forall x\,\forall y\,\forall z\,
      (x=y\land y=z\limp x=z)\\
      \forall x\,\forall y\,(x=y\limp
      t[z:=x]=t[z:=y])\\
      \forall x\,\forall y\,(x=y\limp
      (\phi[z:=x]\liff\phi[z:=y]))\\
    \end{array}$
  \end{itemize}
\end{ejercicio}

\bigbreak
\begin{ejercicio}[Conmutación de cuantificadores]
  --- El objetivo del ejercicio es mostrar la importancia de las
  ``condiciones de frescura'' que restringen el uso de la reglas
  $\forall$-intro.\ y $\exists$-elim.:
  \begin{center}
    $\infer[\text{\small(si $x\notin\FV(\Gamma)$)}]
    {\Gamma\vdash\forall x\,\phi}{\Gamma\vdash\phi}\qquad\qquad
    \infer[\text{\small(si $x\notin\FV(\Gamma,\psi)$)}]
    {\Gamma\vdash\psi}
    {\Gamma\vdash\exists x\,\phi & \Gamma,\phi\vdash\psi}$
  \end{center}
  Para ello, se considera una fórmula $\phi\equiv\phi(x,y)$ que
  depende (al menos) de dos variables~$x$ e~$y$.
  \begin{enumerate}\parskip-.5ex
  \item[(1)] Derivar las siguientes equivalencias:
    \begin{enumerate}\parskip-.5ex
    \item[(a)] $\forall x\,\forall y\,\phi(x,y)~\liff~
      \forall y\,\forall x\,\phi(x,y)$
    \item[(b)] $\exists x\,\exists y\,\phi(x,y)~\liff~
      \exists y\,\exists x\,\phi(x,y)$
    \end{enumerate}
    verificando que las derivaciones respetan las condiciones de
    frescura.
  \item[(2)] Derivar las siguientes implicaciones:
    \begin{enumerate}\parskip-.5ex
    \item[(a)] $\exists x\,\forall y\,\phi(x,y)~\limp~
      \forall y\,\exists x\,\phi(x,y)$\quad
      respetando las condiciones de frescura;
    \item[(b)] $\forall y\,\exists x\,\phi(x,y)~\limp~
      \exists x\,\forall y\,\phi(x,y)$\quad
      sin respetar las condiciones de frescura.
    \end{enumerate}
  \item[(3)] Con contraejemplos en la matemática usual
    (convergencia simple/uniforme, continuidad simple/uniforme, etc.),
    explicar por qué la implicación del ítem (2.b) es falsa.
  \end{enumerate}
\end{ejercicio}

\bigbreak
\begin{ejercicio}[Una función involutiva]
  --- Se trabaja en un vocabulario $\mathcal{V}$ con un símbolo de
  función~$f$ de aridad~$1$, que representa una función definida sobre
  todo el universo del discurso.
  Derivar el teorema que dice que «si~$f$ es involutiva, entonces $f$
    es biyectiva», es decir:
  \begin{center}
    $\forall x\,f(f(x))=x~~\limp~~
    \forall x\,\forall x'\,(f(x)=f(x')\limp x=x')~~\land~~
    \forall y\,\exists x\,f(x)=y\,.$
  \end{center}
\end{ejercicio}

\begin{ejercicio}[Teoría Grp]
	Derivar los siguientes teoremas.
	\begin{enumerate}\parskip-.5ex
		\item $\text{Grp}\vdash \forall x,~\forall y~x*y=y\Ra x=e$
		\item $\text{Grp}\vdash \forall x,~\forall y~y*x=y\Ra x=e$
		\item $\text{Grp}\vdash \forall x,~\exists y~x*y=y\Ra x=e$
		\item $\text{Grp}\vdash \forall x\forall y\forall z,~x*y=e\wedge x*z=e \Ra y=z$
	\end{enumerate}
\end{ejercicio}

\begin{ejercicio}[Aritmética de Peano]
	Demostrar los siguientes teoremas.
	\begin{enumerate}\parskip-.5ex
		\item $\text{PA}\vdash  s(s(0))+ s(s(0))=s(s(s(s(0))))$
		\item $\text{PA}\vdash  s(s(0))\times s(s(0))=s(s(s(s(0))))$
		\item $\text{PA}\vdash \forall y~0+y=y$
		\item $\text{PA}\vdash \forall x\forall y~s(x)+y=s(x+y)$
		\item $\text{PA}\vdash \forall x\forall y~x+y=y+x$
	\end{enumerate}
\end{ejercicio}

\begin{ejercicio}
	Demostrar que la teoría de conjuntos de Frege es inconsistente.
\end{ejercicio}



\end{document}
