\documentclass[a4paper,12pt]{book}
\usepackage{etex}
\usepackage[utf8]{inputenc}
\usepackage[T1]{fontenc}
\usepackage{fullpage}
\usepackage{amsmath}
\usepackage{amsthm}
\usepackage{txfonts}
\usepackage{latexsym}
\usepackage{stmaryrd}
\usepackage{amssymb}
\usepackage{mathrsfs}
%\usepackage[all]{xy}
\usepackage{proof}
\usepackage[sans]{dsfont}
\usepackage[spanish]{babel}


\newcommand{\Ra}{\Rightarrow}
\newcommand{\ra}{\rightarrow}
\newcommand{\N}{\mathbb{N}}
\newcommand{\R}{\mathbb{R}}
\newcommand{\te}{\text}
\newcommand{\Lra}{\Leftrightarrow}
\newcommand{\lra}{\leftrightarrow}

\def\limp{\Rightarrow}
\def\liff{\Leftrightarrow}
\def\ForIn#1#2{(\forall#1\,{\in}\,#2)}
\def\ExIn#1#2{(\exists#1\,{\in}\,#2)}
\def\Pow{\mathfrak{P}}
\def\id{\mathrm{id}}
\def\dom{\mathrm{dom}}
\def\img{\mathrm{img}}
\def\pto{\rightharpoonup}
\def\into{\hookrightarrow}
\def\onto{\twoheadrightarrow}
\def\inonto{\mathbin{\tilde{\to}}}
\def\Pow{\mathfrak{P}}
\def\id{\mathrm{id}}
\def\dom{\mathrm{dom}}
\def\img{\mathrm{img}}
\def\pto{\rightharpoonup}
\def\into{\hookrightarrow}
\def\onto{\twoheadrightarrow}
\def\inonto{\mathbin{\tilde{\to}}}
\def\Z{\mathbb{Z}}
\def\Q{\mathbb{Q}}


%%%
\theoremstyle{definition}
\newtheorem{ejercicio}{Ejercicio}
\outer\long\def\COUIC#1{}
\outer\long\def\Solucion#1{\par
	{\sl\small\noindent\textbf{Solución:}\quad#1\par}}
%%% Comentar la siguiente línea para mostrar las soluciones
\outer\long\def\Solucion#1{}

\begin{document}
	
	\noindent
	\centerline{\sc
		Facultad de Ciencias\hfill---\hfill
		Fundamentos de la Matemática\hfill---\hfill
		Segundo semestre de 2025}\smallbreak\hrule
	
	\bigbreak
	\centerline{\Large\textbf{Práctico 3: Teoría de conjuntos de Zermelo}}
	\bigbreak
	
	Comentario: se pide hacer pruebas en la teoría de conjuntos de Zermelo (sin esquema de remplazo). No hace falta escribir derivaciones (sería demasiado engorroso), pero es bueno entender que se podría hacerlo.
	
	\begin{ejercicio}
		Probar las siguientes afirmaciones:
		\begin{enumerate}\parskip -.5ex
			\item $\{a,b\}=\{c,d\}$ si y s\'olo si $\left ((a=c) \wedge (b=d)\right )\vee \left ((a=d) \wedge (b=c)\right )$,
			\item $\{a\}=\{b\}$ si y s\'olo si $a=b$,
			\item $(a,b)=(c,d)$ si y s\'olo si $(a=c) \wedge (b=d)$
		\end{enumerate}
	\end{ejercicio}
	
	\begin{ejercicio}
		\begin{enumerate}\parskip -.5ex
			\item Probar que la clase de todos los conjuntos no es un conjunto.
			\item Probar que la clase de los conjuntos unitarios (con un solo elemento) no es un conjunto.
			\item Deducir que la clase de todos los conjuntos finitos no es un conjunto (si bien aún no está definido, todo conjunto unitario es finito).
		\end{enumerate}
	\end{ejercicio}
	
	\begin{ejercicio}
		Sea $A$ un conjunto.
		\begin{enumerate}\parskip -.5ex
			\item Probar que $\bigcup\{A\} = A$.
			\item Dada la clase $b=\{\{a\}/a \in A\}$,  escribirla de modo $\{x~:~\phi(x)\}$ y probar que es un conjunto.
			\item Probar que $\bigcup b=A$.
		\end{enumerate}
	\end{ejercicio}
	
	\begin{ejercicio}
	Probar que $\forall x,~\bigcup \mathcal{P} (x)=x\subseteq \mathcal{P} (\bigcup x)$ y dar un ejemplo que pruebe que la inclusi\'on puede ser estricta. 
	\end{ejercicio}

	
	

	
		\paragraph*{Imágenes y preimágenes}
	Dada una función $f:X\to Y$ y dos subconjuntos
	$A\subseteq X$, $B\subseteq Y$, se recuerda que:
	\begin{itemize}\parskip-.5ex
		\item La \emph{imagen} de $A\subseteq X$ por $f$ está definida por\quad
		$f(A):=\{y\in Y:\ExIn{x}{A}f(x)=y\}$\hfill$({\subseteq}~Y)$
		\item La \emph{preimagen} de $B\subseteq Y$ por $f$ está definida por\quad
		$f^{-1}(B):=\{x\in X:f(x)\in B\}$\hfill $({\subseteq}~X)$
	\end{itemize}
	
	\begin{ejercicio}
	Sea una función $f:X\to Y$, con subconjuntos
	$A,A_1,A_2\subseteq X$ y $B,B_1,B_2\subseteq Y$.
	\begin{enumerate}\parskip-.5ex
		\item[(1)] Para cada una de las siguientes igualdades, decir si es
		verdadera (demostrándola) o falsa (dando un contra-ejemplo, e
		indicando si una de la dos inclusiones se cumple):
		$$\begin{array}{c}
			a)~~f(\varnothing)=\varnothing\qquad\qquad
			b)~~f(X)=Y\qquad\qquad c)~~f(A^c)=(f(A))^c\\[3pt]
			d)~~f(A_1\cup A_2)=f(A_1)\cup f(A_2)\qquad\qquad
			e)~~f(A_1\cap A_2)=f(A_1)\cap f(A_2)\\
		\end{array}$$
		\item[(2)] Mismo ejercicio con las siguientes igualdades:
		$$\begin{array}{c}
			a)~~f^{-1}(\varnothing)=\varnothing\qquad\qquad
			b)~~f^{-1}(Y)=X\qquad\qquad c)~~f^{-1}(B^c)=(f^{-1}(B))^c\\[3pt]
			d)~~f^{-1}(B_1\cup B_2)=f^{-1}(B_1)\cup f^{-1}(B_2)\qquad\qquad
			e)~~f^{-1}(B_1\cap B_2)=f^{-1}(B_1)\cap f^{-1}(B_2)\\
		\end{array}$$
		\item[(3)] ¿Qué cambia en 1.\ y 2.\ si se remplaza las uniones e
		intersecciones binarias ($A_1\cup A_2$, $A_1\cap A_2$, etc.) por
		uniones e intersecciones infinitarias ($\bigcup_{i\in I}A_i$,
		$\bigcap_{i\in I}A_i$, etc.)?
	\end{enumerate}
\end{ejercicio}
	
	\begin{ejercicio}
		Sean $f,g$ grafos funcionales. Decimos que $f$ y $g$ son compatibles si se cumple $\forall x \in dom(f)\cap dom(g),~f(x)=g(x)$.
		\begin{enumerate}\parskip -.5ex
			\item Dados dos grafos funcionales $f,g$, \begin{enumerate}\parskip -.5ex
				\item probar que el grafo $f\cup g$ es funcional si y s\'olo si $f$ y $g$ son compatibles. 
				
				\item probar que si $f$ y $g$ son compatibles $dom(f\cup g)=dom(f)\cup dom(g)$ y $img(f\cup g)=img(f)\cup img(g)$.
			\end{enumerate}
			\item Dado un conjunto $C$ de grafos funcionales, probar que el grafo $F=\bigcup C$ es una funci\'on si y s\'olo si los elementos de $C$ son funciones compatibles dos a dos.
		\end{enumerate}
	\end{ejercicio}
	
	\begin{ejercicio}
		Sean $X,Y$ dos conjuntos y $\phi(x,y)$ una fórmula tal que $\forall x\in X\,\exists!y\in Y~\phi(x,y)$. Probar que existe $f:X\to Y$ tal que $\forall x\in X~\phi(x,f(x))$. Concretamente, hay que probar:
		$$\exists f,\quad f:X\to Y~\wedge~\forall x\in X~\phi(x,f(x))
		$$
		usando las definiciones de función, dominio, imagen, etc y los axiomas. Puede ser util que $y=f(x)\Lra (x,y)\in f$.
		
		Observar que quitando la propiedad de unicidad en $y$ el argumento deja de funcionar (se necesita un axioma posterior...).
	\end{ejercicio}
	


	
	\paragraph*{Inyecciones y sobreyecciones}
	Intuitivamente, la existencia de una inyección $f:A\into B$ significa
	que «$A$ es más pequeño que~$B$», y la existencia de una sobreyección
	$f:A\onto B$ que «$A$ es más grande que~$B$».
	Pero las cosas no son tan sencillas...
	
	\begin{ejercicio}[Un problema de elección]
		Sean $A,B$ dos conjuntos cualesquiera.
		\begin{enumerate}
			\item[(1)] Supongamos que $A\neq\varnothing$ y que existe una
			inyección $f:A\into B$.
			Mostrar que existe una sobreyección $g:B\onto A$, y que se puede
			construir~$g$ tal que $g\circ f=\id_A$.
			\item[(2)] ¿Qué pasa cuando $A=\varnothing$?
			\item[(3)] Supongamos que existe una sobreyección $f:A\onto B$.
			¿Existe una inyección $g:B\into A$?
		\end{enumerate}

	\end{ejercicio}
	
	\begin{ejercicio}[Cantor]\label{ejer:Cantor}
		--- Sea~$A$ un conjunto.
		\begin{enumerate}
			\item[(1)] Demostrar que no existe ninguna sobreyección
			$f:A\to\Pow(A)$.\\
			(\textit{Sugerencia}: considerar el conjunto
			$\{x\in A:x\notin f(x)\}$.)
			\item[(2)] Deducir de (1) que no existe ninguna inyección
			$g:\Pow(A)\to A$.
			\item[(3)] Deducir de lo anterior que para todo conjunto~$a$,
			tenemos que $\Pow(\bigcup a)\notin a$.
		\end{enumerate}
	\end{ejercicio}
	
	
	\paragraph{Conjuntos equipotentes}
	Se recuerda que dos conjuntos~$X$ e~$Y$ son \emph{equipotentes} cuando
	existe una biyección $f:X\inonto Y$.
	
	\begin{ejercicio}[Teorema de Cantor-Bernstein-Schröder]%
		\label{ejer:CantorBernstein}
		--- Sea $X$ un conjunto dado con una inyección $h:X\into X$ (que
		define así una biyección entre~$X$ y $h(X)$), e~$Y$ un conjunto tal
		que $h(X)\subseteq Y\subseteq X$.
		Se trata de demostrar que~$Y$ es equipotente a~$X$.
		\begin{enumerate}\parskip-.5ex
			\item[(1)] Demostrar que existe un subconjunto \emph{minimal}
			$Z\subseteq X$ tal que~~$(X-Y)\subseteq Z$~~y~~$h(Z)\subseteq Z$.
			\item[(2)] Demostrar que $h(Z)=Z\cap Y$.
			\item[(3)] Usando la inyección~~$h:X\into X$~~y el
			subconjunto~~$Z\subseteq X$~~definido en (1), construir una
			biyección $h':X\inonto Y$.\hfill
			(\textit{Sugerencia:} definir $h'(x)$ por casos según que
			$x\in Z$ o no.)
			\item[(4)] Deducir de lo anterior el teorema de
			Cantor-Bernstein-Schröder:\\
			Si $A$ y $B$ son dos conjuntos tales que existen inyecciones
			$f:A\into B$ y $g:B\into A$, entonces~$A$ y~$B$ son equipotentes.
		\end{enumerate}
	\end{ejercicio}
	
	El teorema de Cantor-Bernstein-Schröder es una herramienta
	fundamental para demostrar que dos conjuntos~$A$ y~$B$ son
	equipotentes, en la medida en que reduce el problema (difícil) de la
	construcción de una biyección entre~$A$ y~$B$ al problema (más
	sencillo) de la construcción de dos inyecciones $f:A\into B$ y
	$g:B\into A$ sin relación entre ellas.
	
	\begin{ejercicio}[«Lo veo pero no lo creo»]
		Usando el teorema anterior, demostrar que:
		\begin{enumerate}
			\item[(1)] Los conjuntos $\R^2$ (el «plano») y $\R$ (la «recta»)
			son equipotentes.\\
			(Sugerencia: remplazar $\R$ por el intervalo abierto $(0,1)$.)
			\item[(2)] Los conjuntos $\Pow(\N)$ y $\R$ son equipotentes.
		\end{enumerate}
		(\emph{Observación histórica: la existencia de una biyección entre
			el plano y la recta fue descubierta en 1877 por Cantor, que
			escribió en una carta a Dedekind: «lo veo pero no lo creo».})
	\end{ejercicio}
	
	\paragraph*{Conjuntos Dedekind-infinitos}
	Un conjunto $X$ es \emph{Dedekind-infinito} (o \emph{D-infinito})
	cuando existe una función $f:X\to X$ inyectiva pero no sobreyectiva.
	
	\begin{ejercicio}[Conjuntos D-infinitos]~\par
		\begin{enumerate}
			\item[(1)] Verificar que los conjuntos $\N$, $\Z$, $\Q$ y $\R$ son
			D-infinitos.
			\item[(2)] Mostrar que si $X\subseteq Y$ y $X$ es D-infinito,
			entonces $Y$ también es D-infinito.
			\item[(3)] Sea $X$ un conjunto D-infinito dado con una inyección
			$f:X\into X$ y un elemento $a\in X$ tal que $a\notin\img(f)$.
			Mostrar que existe un subconjunto \emph{minimal} $N\subseteq X$
			tal que $a\in N$ y $f(N)\subseteq N$.
			Deducir que el conjunto $N$ satisface el principio de inducción:
			\begin{quote}
				Si $P\subseteq N$ es tal que $a\in P$ y
				$f(P)\subseteq P$, entonces $P=N$.
			\end{quote}
			(Intuitivamente, $N$ es una copia de $\N$ adentro de $X$, donde
			$a$ y $f$ tienen el papel de $0$ y de la función sucesor.
			En particular, el conjunto $N$ equipado con $a$ y $f$ es isomorfo
			a $\N$.)
		\end{enumerate}
	\end{ejercicio}
	
\end{document}
