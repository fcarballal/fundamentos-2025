\documentclass[a4paper, 12pt]{report}


\usepackage[utf8]{inputenc}
\usepackage[T1]{fontenc}
\usepackage[spanish]{babel} 
\usepackage{graphicx}
\usepackage{amsmath}
\usepackage{amssymb}
\usepackage{mathtools}
\usepackage{amsthm}
\usepackage{bbm}
%\usepackage[shortlabels]{enumitem}
\usepackage{enumerate}
\usepackage{array,tabularx}
\usepackage{float}
\usepackage{wrapfig}
\usepackage[export]{adjustbox}
\usepackage[rightcaption]{sidecap}
\usepackage{multirow}
\usepackage{subfig}
\usepackage{capt-of}
\usepackage{captdef}
\usepackage{color}
\usepackage{ebproof}
\usepackage{soul}
\usepackage{hyperref}
\usepackage{ebproof}
\usepackage{stmaryrd}
\usepackage{fullpage}
\usepackage{xcolor}

\newcommand{\Ra}{\Rightarrow}
\newcommand{\ra}{\rightarrow}
\newcommand{\N}{\mathbb{N}}
\newcommand{\R}{\mathbb{R}}
\newcommand{\te}{\text}
\newcommand{\Lra}{\Leftrightarrow}
\newcommand{\lra}{\leftrightarrow}

\begin{document}

\centerline{\Huge\bf Apuntes de Introducción a la Computación}



\tableofcontents

\chapter{Introducción}

 En este curso planteo a grandes rasgos tres objetivos. El primero es que aprendan las bases de la programación. 
Otro objetivo es que desarrollen la capacidad de utilizar las computadoras con fines matemáticos, como por ejemplo para determinar si un número es primo, o aproximar numéricamente el valor de una integral, o generar gráficas de distintos tipos.
El otro objetivo es que desarrollen la capacidad de entender la computación como un concepto abstracto matemático, para desarrollar programas en base a razonamiento matemático y además poder demostrar matemáticamente propiedades sobre estos programas.
De modo sintético, este curso se trata de programación, de computación para matemática y de matemática para computación.

Se espera que los contenidos de este capítulo aporten intuiciones que ayuden a comprender la materia.



\section{Conceptos básicos de computación e intuiciones}

La computación se trata de operaciones definidas por reglas precisas que operan con objetos finitos y en una cantidad finita de pasos llegan a un resultado.

Los objetos finitos, a los que en general llamaremos {\bf datos}, pueden ser por ejemplo números naturales, palabras o listas finitas. Cabe aclarar que al decir que los números naturales son finitos nos referimos a que cada número natural es un objeto finito. Por supuesto que el conjunto de todos los números naturales, por otra parte, es infinito.

A una operación definida por reglas precisas que opera con datos y termina en finitos pasos le llamaremos {\bf algoritmo}. Un programa es lo mismo que un algoritmo. Un ejemplo de algoritmo es el de la suma que se aprende en la escuela. Dados dos números, aplicando una secuencia finita de pasos bien definida se llega al resultado.


Una buena analogía consta de las recetas de cocina. Sin embargo, hay una diferencia importante. En las recetas de cocina, por la naturaleza de la situación, pueden haber ciertas ambigüedades, o puntos en los que hay que tomar una decisión, de modo que si dos personas realizan la misma receta el resultado puede ser distinto, sin que ninguno de los dos lo haya hecho mal. Pueden haber frases como <<un poquito de sal>> que según la persona se traduzcan a distintas cantidades. En computación, un algoritmo bien ejecutado debe tener un único resultado correcto, como ocurre con la suma. Que las reglas sean precisas significa que no hay margen de variabilidad en lo que se debe hacer, ni puntos en los que un ser con voluntad propia deba pensar y tomar una decisión. Esto es importante por dos motivos. Primero, asegura que si se realiza por una persona, hay un único resultado correcto. Segundo (y tal vez más importante), es necesario para que pueda ser realizado automáticamente por una computadora, que en lugar de un ser con pensamiento y voluntad es una cosa.

Como los algoritmos deben estar bien definidos y no puede haber ninguna ambigüedad, son en esencia conceptos matemáticos. De hecho, la computación teórica es una rama de la matemática.
\subsection{Bases de la formalización matemática}

Una forma de representar algoritmos en computación teórica es mediante funciones $f:\N\to\N$, donde la idea es que si $n$ es la entrada del programa, entonces $f(n)$ es la salida. Como se trata de un algoritmo, no puede ser cualquier función, sino una que puede calcularse en finitos pasos con reglas precisas. A primera vista puede parecer que usar solamente números naturales es una restricción, pero de hecho alcanza para hablar de todo lo que se puede programar. De hecho, dentro de la computadora todos los distintos tipos de datos  se representan con secuencias de bits (\emph{binary digits}, es decir dígitos binarios: 0 o 1) y cualquier secuencia de bits se puede interpretar también como un número escrito en binario. Por lo tanto, todos los datos finitos se pueden representar con números (en general muy grandes). Debido a esto, un programa que recibe como entrada una imagen y retorna la imagen con alguna modificación, por ejemplo, se puede ver como una función $f:\N\to\N$ que recibe el número que codifica la imagen original y retorna el número que codifica la imagen modificada.

Hay distintos formalismos que permiten definir funciones $f:\N\to\N$ computables, es decir, que corresponden a algoritmos. Algunos de ellos son las máquinas de Turing, el cálculo lambda y la teoría de las funciones recursivas. Notablemente, aunque todos son formalismos muy distintos, siempre dan lugar a exactamente el mismo conjunto de funciones computables. Este conjunto de funciones computables no es todo $\N^\N$. De hecho, el conjunto de las funciones computables es numerable, por lo que la mayoría de las funciones no lo son.

Por último, está la pregunta de si la definición matemática de función computable corresponde con las funciones que en el mundo físico se pueden calcular con una computadora. La Tesis de Church-Turing conjetura que efectivamente es el caso. Al ser algo del mundo físico y salirse de la matemática, no hay ninguna prueba formal, pero en general es asumido como cierto.


\section{Arquitectura de Von Neumann}\label{sec-ArqVonNeu}


\end{document}



La teoría de las funciones recursivas lista ciertas reglas para construir funciones y define como funciones recursivas todas las que se pueden construir a partir de esas reglas. Esto es importante porque de hecho no todas las funciones son recursivas (hay funciones $f:\N\to\N$ que no se pueden construir con las reglas y de hecho son funciones que no se pueden calcular con una computadora).

